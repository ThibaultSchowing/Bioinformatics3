\documentclass[10pt,a4paper]{article}
\usepackage[utf8]{inputenc}

% \usepackage{ngerman}  % german documents
\usepackage{graphicx}  % import graphics einbinden
\usepackage{listings}  % support source code listing
\usepackage{amsmath}  % math stuff
\usepackage{amssymb} % 
\usepackage{a4wide} % wide pages
\usepackage{fancyhdr} % nice headers
\usepackage{float}
\usepackage{longtable}
\usepackage{xcolor}
\usepackage{booktabs}
\definecolor{darkpastelgreen}{rgb}{0.01, 0.75, 0.24}
\definecolor{spirodiscoball}{rgb}{0.06, 0.75, 0.99}
\definecolor{smalt}{rgb}{0.0, 0.2, 0.6}
\definecolor{armygreen}{rgb}{0.29, 0.33, 0.13}
\definecolor{awesome}{rgb}{1.0, 0.13, 0.32}
\definecolor{bittersweet}{rgb}{1.0, 0.44, 0.37}
\definecolor{bananayellow}{rgb}{1.0, 0.88, 0.21}
\definecolor{blue}{rgb}{0.0, 0.0, 1.0}
\definecolor{red}{rgb}{1.0, 0.0, 0.0}
\definecolor{green}{rgb}{0.0, 1.0, 0.0}

% for multiple figures
\usepackage{subcaption}


\lstset{basicstyle=\footnotesize,language=Python,breaklines=true,numbers=left, numberstyle=\tiny, stepnumber=5,firstnumber=0, numbersep=5pt} % set up listings
\pagestyle{fancy}             % header
\setlength{\parindent}{0pt}   % no indentation
 

\usepackage[pdfpagemode=None, colorlinks=true,  % url coloring
           linkcolor=blue, urlcolor=blue, citecolor=blue, plainpages=false, 
           pdfpagelabels,unicode]{hyperref}
           
% change enums style: first level (a), (b), (c)           
\renewcommand{\labelenumi}{(\alph{enumi})}
\renewcommand{\labelenumii}{(\arabic{enumii})}

\newcommand{\norm}[1]{\left\lVert#1\right\rVert}

%lecture name
\newcommand{\lecture}{
	Bioinformatics III
}           

%assignment iteration
\newcommand{\assignment}{
	Ninth Assignment
}


%set up names, matricle number, and email
\newcommand{\authors}{
  \begin{tabular}{rl}
    \href{mailto:s8tbscho@stud.uni-saarland.de}{Thibault Schowing} & (2571837)\\
    \href{mailto:wiebkeschmitt@outlook.de}{Wiebke Schmitt} & (2543675)
  \end{tabular}
}

% use to start a new exercise
\newcommand{\exercise}[1]
{
  \stepcounter{subsection}
  \subsection*{Exercise \thesubsection: #1}

}

\begin{document}
\title{\Large \lecture \\ \textbf{\normalsize \assignment}}
\author{\authors}

\setlength \headheight{25pt}
\fancyhead[R]{\begin{tabular}{r}\lecture \\ \assignment \end{tabular}}
\fancyhead[L]{\authors}


\setcounter{section}{9} % modify for later sheets, i.e. 2, 3, ...
%\section{Introduction to Python and some Network Properties} % optional, note that section invocation sets the section counter + 1, so adapt the setcounter command
\maketitle

%EXERCICE 1
\exercise{Extreme Pathways and Steady State Flux Distribution. Paper-based}

\begin{enumerate}

% A
\item \textbf{Construct the stoichiometric matrix:}\\ 

%\lstinputlisting[label=qtew-1, caption={Data Matrix class script}] {../Scripts/data\string_matrix.py}


\begin{figure}[H]
	\centering
	\includegraphics[width=0.7\linewidth]{img/ex1}
	\caption{Reaction network to derive extreme pathways from.}
	\label{fig:ex1}
\end{figure}


\begin{table}[H]
	\centering
	\caption{Stochiometric Matrix}
	\label{stociometric1}
	\begin{tabular}{|l|l|l|l|l|l|l|l|l|l|l|l|l|}
		\hline
		& v1 & v2 & v3 & v4 & v5 & v6 & v6 & v7 & b1 & b2 & b3 & b4 \\ \hline
		A & -1 & 0  & 0  & 0  & 0  & 0  & 0  & 0  & 1  & 0  & 0  & 0  \\ \hline
		B & -2 & 0  & 0  & 0  & 0  & 0  & 0  & 0  & 0  & 1  & 0  & 0  \\ \hline
		C & 2  & 0  & 2  & 0  & 2  & 0  & 0  & -2 & 0  & 0  & 0  & 0  \\ \hline
		D & 0  & 0  & 0  & 1  & -1 & 1  & -1 & 0  & 0  & 0  & 0  & 0  \\ \hline
		E & 0  & 0  & 0  & 0  & 0  & -1 & 1  & 1  & 0  & 0  & 0  & -1 \\ \hline
		F & 0  & -1 & 0  & 0  & 0  & 0  & 0  & 0  & 0  & 0  & 1  & 0  \\ \hline
		G & 0  & 3  & -3 & -3 & 0  & 0  & 0  & 0  & 0  & 0  & 0  & 0  \\ \hline
	\end{tabular}
\end{table}



%B
\item \textbf{Calculate from the stoichiometric matrix the extreme pathways. Give pathways as formulas.} \\

\item \textbf{Formulate the pathway length matrix. Which information does it provide (diagonal vs off-diagonal entries?}\\

\item \textbf{Formulate the reaction particiaption matrix. Which information does it provide?}\\

\item \textbf{Cut-set.} \textit{A reaction or a set of reactions are essential for the network, when there is no output if this reactions are blocked. List all those reactions.}\\

\item \textbf{Biomass producation.}\textit{Now assume that the potential input into the network through b1, b2, and b3, i.e., the sum of the fluxes through these reactions is limited to 5 units. How must this input be distributed onto these reactions to give the highest output through b4?}\\





\end{enumerate}


%\newpage
%\lstinputlisting[label=qtew-1, caption={boolean\textunderscore network.py}] {../Scripts/boolean\string_network.py}
%\newpage
%\lstinputlisting[label=qtew-1, caption={Main function that tests the boolean network}] {../Scripts/main\string_boolean.py}




% NEW EXERCICE
\newpage
\exercise{Hands-on with COnstraint-Based Reconstruction and Analysis (COBRA) in Python.}

\begin{enumerate}
	
	\item \textbf{Provide formulas of the reactions participating in the chain.}
	
	\lstinputlisting[language={}]{../Scripts/Output.txt}
	
	
	
	\item \textbf{Fill in the stoichiometry matrix.}
	
	\begin{table}[H]
		\centering
		\caption{Stochiometry matrix}
		\label{my-label}
		\begin{tabular}{|l|l|l|l|l|l|l|l|l|l|l|}
			\hline
			& HEX1 & PGI & PFK & FBA & TPI & GAPD & PGK & PGM & ENO & PYK \\ \hline
			ATP   & -1   &     & -1  &     &     &      & -1  &     &     & 1   \\ \hline
			GLC   & -1   &     &     &     &     &      &     &     &     &     \\ \hline
			ADP   & 1    &     & 1   &     &     &      & 1   &     &     & -1  \\ \hline
			G6P   & 1    & -1  &     &     &     &      &     &     &     &     \\ \hline
			H     & 1    &     & 1   &     &     & 1    &     &     &     & -1  \\ \hline
			F6P   &      & 1   & -1  &     &     &      &     &     &     &     \\ \hline
			FDP   &      &     & 1   & -1  &     &      &     &     &     &     \\ \hline
			DHAP  &      &     &     & 1   & -1  &      &     &     &     &     \\ \hline
			G3P   &      &     &     & 1   & 1   & -1   &     &     &     &     \\ \hline
			NAD   &      &     &     &     &     & -1   &     &     &     &     \\ \hline
			PI    &      &     &     &     &     & -1   &     &     &     &     \\ \hline
			13DPG &      &     &     &     &     & 1    & 1   &     &     &     \\ \hline
			NADH  &      &     &     &     &     & 1    &     &     &     &     \\ \hline
			3PG   &      &     &     &     &     &      & -1  & 1   &     &     \\ \hline
			2PG   &      &     &     &     &     &      &     & -1  & -1  &     \\ \hline
			PEP   &      &     &     &     &     &      &     &     & 1   & -1  \\ \hline
			H2O   &      &     &     &     &     &      &     &     & 1   &     \\ \hline
			PYR   &      &     &     &     &     &      &     &     &     & 1   \\ \hline
		\end{tabular}
	\end{table}
	
	
	
	\item \textbf{Create the model for the given chain of reactions. Provide the number of reactions, metabolites and genes in it. (Python) }\\
	
	We obtain 10 reactions implying 16 genes and 18 metabolites. Code below.
	
	
	
	\lstinputlisting[label=lsdddt-1, caption={Correlation network}] {../Scripts/Assignment9.py}
	
	
\end{enumerate}



% NEW EXERCICE
\newpage
\exercise{FFEK Algorithm}
\begin{enumerate}
	
	% A
	\item \textbf{Apply the Ford, Fulkerson, Edmonds, and Karp (FFEK) algorithm explained in the lecture to
		determine the s-t-cut and the capacity of the network given below.
		For each iteration, give the indices of the nodes, the resulting f-augmenting path with its capacity,
		and the updated val(f). Sketch the newly found f-augmenting paths. Also update the currents
		through the arcs.
		If you find multiple possible paths from s to t with the same length, then choose the one with the
		highest $\Delta q$}\\
	
	%\lstinputlisting[label=lsdddt-1, caption={Correlation network}] {../Scripts/network.py}
	
	
	\textbf{FFEK iterations\footnote{Helpful video: \url{youtube.com/watch?v=GiN3jRdgxU4}}} For each iteration, the displayed graph is the result after applying the FFEK algorithm. The traces of the algorithm are stored in the queue (net nodes to visit), the Visited nodes and the trace which indicates from which node a node has been discovered. The "current" list store the order of the nodes on which the algorithm is applied and the resulting path and flow are also marked in the table. The figure \ref{fig:fuckingflownetmaxflow} shows the final flows and capacity.\\
	
	The max flow here is 11 and so is the min-cut. For example, we can split the vertex as \{S, a, b, c, d, e, f \} and \{g, h, i, j, k, t\} and the value of the cut is $ 5 + 2 + 4 = 11 $.
	
	
	
	
\begin{figure}[H]
	\centering
	\includegraphics[width=0.9\linewidth]{img/fuckingFlowNetiter1}
	\caption{Iteration 1}
	\label{fig:fuckingflownetiter1}
\end{figure}

	
	
\begin{figure}[H]
	\centering
	\includegraphics[width=0.9\linewidth]{img/fuckingFlowNetiter2e}
	\caption{Iteration 2}
	\label{fig:fuckingflownetiter2e}
\end{figure}
	
	
	
\begin{figure}[H]
	\centering
	\includegraphics[width=0.9\linewidth]{img/fuckingFlowNetiter3}
	\caption{Iteration 3}
	\label{fig:fuckingflownetiter3}
\end{figure}
	
	
	
	
\begin{figure}[H]
	\centering
	\includegraphics[width=0.9\linewidth]{img/fuckingFlowNetiter4}
	\caption{Iteration 4}
	\label{fig:fuckingflownetiter4}
\end{figure}
	
	
	
	
\begin{figure}[H]
	\centering
	\includegraphics[width=0.9\linewidth]{img/fuckingFlowNetiter5}
	\caption{Iteration 5 - Final iteration}
	\label{fig:fuckingflownetiter5}
\end{figure}


\begin{figure}[H]
	\centering
	\includegraphics[width=0.9\linewidth]{img/fuckingFlowNetMaxFlow}
	\caption{Resulting maximum flow}
	\label{fig:fuckingflownetmaxflow}
\end{figure}



\end{enumerate}


\end{document}