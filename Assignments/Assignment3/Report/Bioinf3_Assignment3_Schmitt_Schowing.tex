\documentclass[10pt,a4paper]{article}
\usepackage[utf8]{inputenc}

% \usepackage{ngerman}  % german documents
\usepackage{graphicx}  % import graphics einbinden
\usepackage{listings}  % support source code listing
\usepackage{amsmath}  % math stuff
\usepackage{amssymb} % 
\usepackage{a4wide} % wide pages
\usepackage{fancyhdr} % nice headers
\usepackage{float}
\lstset{basicstyle=\footnotesize,language=Python,breaklines=true,numbers=left, numberstyle=\tiny, stepnumber=5,firstnumber=0, numbersep=5pt} % set up listings
\pagestyle{fancy}             % header
\setlength{\parindent}{0pt}   % no indentation

\usepackage[pdfpagemode=None, colorlinks=true,  % url coloring
           linkcolor=blue, urlcolor=blue, citecolor=blue, plainpages=false, 
           pdfpagelabels,unicode]{hyperref}
           
% change enums style: first level (a), (b), (c)           
\renewcommand{\labelenumi}{(\alph{enumi})}
\renewcommand{\labelenumii}{(\arabic{enumii})}

%lecture name
\newcommand{\lecture}{
	Bioinformatics III
}           

%assignment iteration
\newcommand{\assignment}{
	Third Assignment
}


%set up names, matricle number, and email
\newcommand{\authors}{
  \begin{tabular}{rl}
    \href{mailto:s8tbscho@stud.uni-saarland.de}{Thibault Schowing} & (2571837)\\
    \href{mailto:wiebkeschmitt@outlook.de}{Wiebke Schmitt} & (2543675)
  \end{tabular}
}

% use to start a new exercise
\newcommand{\exercise}[1]
{
  \stepcounter{subsection}
  \subsection*{Exercise \thesubsection: #1}

}

\begin{document}
\title{\Large \lecture \\ \textbf{\normalsize \assignment}}
\author{\authors}

\setlength \headheight{25pt}
\fancyhead[R]{\begin{tabular}{r}\lecture \\ \assignment \end{tabular}}
\fancyhead[L]{\authors}


\setcounter{section}{3} % modify for later sheets, i.e. 2, 3, ...
%\section{Introduction to Python and some Network Properties} % optional, note that section invocation sets the section counter + 1, so adapt the setcounter command
\maketitle

%WIEBKE !!! READ THIS !!! 
% Usefull with Latex and maths: a list of the symbols ! https://reu.dimacs.rutgers.edu/Symbols.pdf

%EXERCICE 1
\exercise{}
\begin{enumerate}

% A
\item \textit{Given the states of the features, you want to infer if two proteins are likely to physically
	interact. In practice, log-likelihood ratios are used in binary classification:}\\

\[ 
	log\frac{P(C|S)}{P(\bar{C}|S)} 
\]

\textit{Derive a term that uses observable probabilities such as $ P(S_i|C) $ to calculate the loglikelihood
	ratio from training data. How does the actual classification work?}\\



First we have: 

\[ P(S_i | C) = \frac{P(C|S_i)P(S_i)}{P(C)} \]

And: 

\[ P(S_i | \bar{C}) = \frac{P(\bar{C}|S_i)P(S_i)}{P(\bar{C})} \]

Then we develop the desired final output

\[ \frac{P(S_i|C)}{P(S_i |\bar{C})}  \Longleftrightarrow \frac{P(S_i | C)P(C)}{P(S_i |\bar{C})P(\bar{C})} = \frac{P(C|S_i)P(S_i)}{P(\bar{C}|S_i)P(S_i)} = \frac{P(C|S_i )}{P(\bar{C}|S_i)}\]



\[ log\frac{P(C|S)}{P(\bar{C}|S)} = log \prod_{i}^{n} \frac{P(S_i | C)P(C)}{P(S_i |\bar{C})P(\bar{C})} = \sum_{i}^{n} log \frac{P(S_i | C)P(C)}{P(S_i |\bar{C})P(\bar{C})} = \bigwedge(C|S) \]




% B
\item \textit{bla instructions}\\




% C
\newpage
\item \textit{}\\


%\lstinputlisting[label=ex2-b, caption={ScaleFreeTest.py}] {../Scripts/ScaleFreeTest.py}


\end{enumerate}



% NEW EXERCICE
\newpage
\exercise{Classify real-world network examples}
\begin{enumerate}
	\item 
	
	
	\item 
	
	
	\item 

\end{enumerate}

\end{document}