\documentclass[10pt,a4paper]{article}
\usepackage[utf8]{inputenc}

% \usepackage{ngerman}  % german documents
\usepackage{graphicx}  % import graphics einbinden
\usepackage{listings}  % support source code listing
\usepackage{amsmath}  % math stuff
\usepackage{amssymb} % 
\usepackage{a4wide} % wide pages
\usepackage{fancyhdr} % nice headers
\usepackage{float}
\usepackage{longtable}
\usepackage{xcolor}
\definecolor{darkpastelgreen}{rgb}{0.01, 0.75, 0.24}
\definecolor{spirodiscoball}{rgb}{0.06, 0.75, 0.99}
\definecolor{smalt}{rgb}{0.0, 0.2, 0.6}

\lstset{basicstyle=\footnotesize,language=Python,breaklines=true,numbers=left, numberstyle=\tiny, stepnumber=5,firstnumber=0, numbersep=5pt} % set up listings
\pagestyle{fancy}             % header
\setlength{\parindent}{0pt}   % no indentation

\usepackage[pdfpagemode=None, colorlinks=true,  % url coloring
           linkcolor=blue, urlcolor=blue, citecolor=blue, plainpages=false, 
           pdfpagelabels,unicode]{hyperref}
           
% change enums style: first level (a), (b), (c)           
\renewcommand{\labelenumi}{(\alph{enumi})}
\renewcommand{\labelenumii}{(\arabic{enumii})}

%lecture name
\newcommand{\lecture}{
	Bioinformatics III
}           

%assignment iteration
\newcommand{\assignment}{
	Third Assignment
}


%set up names, matricle number, and email
\newcommand{\authors}{
  \begin{tabular}{rl}
    \href{mailto:s8tbscho@stud.uni-saarland.de}{Thibault Schowing} & (2571837)\\
    \href{mailto:wiebkeschmitt@outlook.de}{Wiebke Schmitt} & (2543675)
  \end{tabular}
}

% use to start a new exercise
\newcommand{\exercise}[1]
{
  \stepcounter{subsection}
  \subsection*{Exercise \thesubsection: #1}

}

\begin{document}
\title{\Large \lecture \\ \textbf{\normalsize \assignment}}
\author{\authors}

\setlength \headheight{25pt}
\fancyhead[R]{\begin{tabular}{r}\lecture \\ \assignment \end{tabular}}
\fancyhead[L]{\authors}


\setcounter{section}{3} % modify for later sheets, i.e. 2, 3, ...
%\section{Introduction to Python and some Network Properties} % optional, note that section invocation sets the section counter + 1, so adapt the setcounter command
\maketitle

%WIEBKE !!! READ THIS !!! 
% Usefull with Latex and maths: a list of the symbols ! https://reu.dimacs.rutgers.edu/Symbols.pdf

%EXERCICE 1
\exercise{}
\begin{enumerate}

% A
\item \textit{Given the states of the features, you want to infer if two proteins are likely to physically
	interact. In practice, log-likelihood ratios are used in binary classification:}\\

\[ 
	log\frac{P(C|S)}{P(\bar{C}|S)} 
\]

\textit{Derive a term that uses observable probabilities such as $ P(S_i|C) $ to calculate the loglikelihood
	ratio from training data. How does the actual classification work?}\\



First we have: 

\[ P(S_i | C) = \frac{P(C|S_i)P(S_i)}{P(C)} \]

And: 

\[ P(S_i | \bar{C}) = \frac{P(\bar{C}|S_i)P(S_i)}{P(\bar{C})} \]

Then we develop the desired final output

\[ \frac{P(S_i|C)}{P(S_i |\bar{C})}  \Longleftrightarrow \frac{P(S_i | C)P(C)}{P(S_i |\bar{C})P(\bar{C})} = \frac{P(C|S_i)P(S_i)}{P(\bar{C}|S_i)P(S_i)} = \frac{P(C|S_i )}{P(\bar{C}|S_i)}\]



\[ log\frac{P(C|S)}{P(\bar{C}|S)} = log \prod_{i}^{n} \frac{P(S_i | C)P(C)}{P(S_i |\bar{C})P(\bar{C})} = \sum_{i}^{n} log \frac{P(S_i | C)P(C)}{P(S_i |\bar{C})P(\bar{C})} = \Lambda(C|S) \]

\[ O(C|S) =  \Lambda(C|S) O(C)\]

The posterior odd is calculated by the odds of an event ($ \frac{p(event)}{1-p(event)}$) multiplied by the likelihood of that event\footnote{Slides V4 - 4}. \\

To do the classification, we must interate through the data and calculate all the priors and likelihood. The prior $P(C)$ is made from an educated guess 
%TODO understand and complete


% B
\item \textit{Shortly discuss: What are the practical advantages of the logarithm and the likelihood ratio
	within this framework? State two reasons why this particular type of classifier may perform
	poorly on a real world dataset.}\\

% https://academic.oup.com/aje/article/153/12/1222/124010
% http://slideplayer.com/slide/5178261/
The logarithm increase is a monotonically increasing function of $x$ hence, for any positive value the maximum value of a function $f(x)$, the maximum of $f(x)$ is equal to the maximum of $log(f(x))$. This simplifies the calculation because we don't need the second derivative. A likelihood function is not concave but the log-likelihood is. Also, as seen in part A, with the log-likelihood we can turn a log of products into a sum of logs. The main inconvenient is that this method assume that all the features are independent and do not take in account the eventual correlations between them. 




% C
\newpage
\item \textit{Use the file training1.tsv to build a model. This basically means to determine all necessary
	priors and likelihoods from part (a). The file layout is explained in README.txt. Report
	$P(C)$ and $P(\bar{C})$ as well as the ten $S_i$ (feature number, variant and log-ratio) with the highest
	absolute log-likelihood ratios. Examine and comment on the results of the training-phase.
	Which features seem to be the most helpful?
}\\

Prior probability $P(C) = 0.78$\\
Prior probability $P(\bar{C}) = 0.22$\\

 



\begin{table}[H]
	\centering
	\caption{10 $S_i$ with the highest absolute log-likelihood ratio}
	\label{10most}
	\begin{tabular}{|l|l|l|}
		\textbf{Feature} & \textbf{Variant} & \textbf{log-ratio} \\
		33 & 0 & -3.7214026458194964 \\
		11 & 3 & -2.565631943311438  \\
		87 & 1 & -2.4686396773241284 \\
		53 & 1 & -2.3351082846996056 \\
		99 & 1 & -2.3061207478263537 \\
		59 & 1 & -2.2779498708596573 \\
		80 & 2 & -2.2779498708596573 \\
		86 & 3 & -2.2550655770260692 \\
		91 & 3 & -2.2173252490432223 \\
		97 & 1 & -2.2099451417455995
	\end{tabular}
\end{table}


%TODO explain and verify please !!! I'm lost !!!
The log likelihood ratio explains that if you have the variant X you have log(likelohood) more chance to have a complex (C). If negative, it diminishes this chance. To interpret the results above, if for the feature 33 you have variant 0, you have 3.7 more chance that it doesn't make a complex than with variant 1,2 or 3. So here the feature 33 is the most helpful.

\lstinputlisting[label=ex2-b, caption={bayes.py}] {../Scripts/bayes.py}

For the train and test 1 and 2 we obtain only 0's as a prediction which is not satisfying according to the real outputs. % Maybe we just don't get how to test the created model. 

\end{enumerate}


% NEW EXERCICE
\newpage
\exercise{Classify real-world network examples}
\begin{enumerate}
	\item \textit{If one of the nodes has a degree of 1, then $\tilde{C}_{i,j}^{(3)} $ is infinite. What is the maximal finite value that the edge-clustering coefficient can take? For which configuration does this occur? Give an example!}
	
	The edge-clustering coefficient cannot be more than 2. In the layout below we see that the clustering coefficient for the link between node 2 and 3 is equal to $\tilde{C}_{2,3}^{(3)} = \frac{1 + 1}{min(2 - 1, 6 - 1)} = \frac{2}{1}$. We see that no matter the degree of node 3, if we connect more node to node 2 the coefficient will decrease even if the number of possible triplet increase. 
	
	
\begin{figure}[H]
	\centering
	\includegraphics[width=0.5\linewidth]{expl}
	\caption{}
	\label{fig:expl}
\end{figure}
	
	
	\item (1)\textit{Give the links that you deleted from the network in (iii) by printing the names of the
		two nodes and their current edge-clustering coefficient in the order of their deletion. Of
		course, add the output to the PDF/sheet that you hand in. Implement this part as a
		script or class-based, there are no specifications you need to adjust to.}
	
	\begin{table}[H]
		\centering
		\caption{Output - Order of removed links}
		\label{table_links}
		\begin{tabular}{lll}
			Tyrion  & Sansa    & 0.33$\bar{3}$  \\
			Joffrey & Hound    & 0.5    \\
			Eddard  & Robert   & 0.5                 \\
			Eddard  & Jon      & 0.5                 \\
			Joffrey & Jaime    & 1.0                 \\
			Hound   & Mountain & 1.0                 \\
			Cersei  & Tyrion   & 1.0                 \\
			Jaime   & Cersei   & 1.0                 \\
			Catelyn & Baelish  & 1.0                 \\
			Sansa   & Baelish  & 1.0                 \\
			Eddard  & Catelyn  & 1.5                 \\
			Sansa   & Arya     & 1.5                 \\
			Eddard  & Sansa    & 1.0                 \\
			Catelyn & Arya     & 1.0                 \\
			Joffrey & Cersei   & 2.0                 \\
			Cersei  & Robert   & 1.0                 \\
			Samwell & Jon      & 2.0                 \\
			Joffrey & Robert   & $\infty$ \\
			Shae    & Tyrion   & $\infty$ \\
			Eddard  & Arya     & $\infty$ \\
			Hound   & Arya     & $\infty$ \\
			Varys   & Baelish  & $\infty$ \\
			Jaime   & Tyrion   & $\infty$ \\
			Cersei  & Mountain & $\infty$ \\
			Catelyn & Sansa    & $\infty$ \\
			Samwell & Jeor     & $\infty$ \\
			Jon     & Jeor     & $\infty$
		\end{tabular}
	\end{table}
	
	Implementation of the network decomposition:  
	
	\lstinputlisting[label=exnet-b, caption={testNetwork.py: decompose the network, output in table \ref{table_links}}] {../Scripts/part2/testNetwork.py}
	
	Implementation of the other classes. New functions have been added directly in the Network classes. 
	
	\lstinputlisting[label=exnet-b, caption={Node.py}] {../Scripts/part2/Node.py}
	\lstinputlisting[label=exnet-b, caption={AbstractNetwork.py}] {../Scripts/part2/AbstractNetwork.py}
	\lstinputlisting[label=exnet-b, caption={GenericNetwork.py}] {../Scripts/part2/GenericNetwork.py}
	
	(2) \textit{Use the links deleted in (1) in reverse order, i.e., the link that was deleted last is now
		used first to construct the communities.}
	
\begin{table}[H]
	\centering
	\caption{Inclusion of links, "x" means two subgraphs have been merged.}
	\label{tabitab}
	\begin{tabular}{|l|l|l|}
		\hline
		Link              & Merge & Graph                                                                                                                                                                                                                                                                                                                                                                                                                                                                                                \\ \hline
		Jon - Jeor        &       & {[}(Jon - Jeor){]}                                                                                                                                                                                                                                                                                                                                                                                                                                                                                   \\ \hline
		Samwell - Jeor    &       & {[}(Jon - Jeor),(Samwell - Jeor){]}                                                                                                                                                                                                                                                                                                                                                                                                                                                                  \\ \hline
		Catelyn - Sansa   &       & {[}(Catelyn - Sansa){]}                                                                                                                                                                                                                                                                                                                                                                                                                                                                              \\ \hline
		Cersei - Mountain &       & {[}(Cersei - Mountain){]}                                                                                                                                                                                                                                                                                                                                                                                                                                                                            \\ \hline
		Jaime - Tyrion    &       & {[}(Jaime - Tyrion){]}                                                                                                                                                                                                                                                                                                                                                                                                                                                                               \\ \hline
		Varys - Baelish   &       & {[}(Varys - Baelish){]}                                                                                                                                                                                                                                                                                                                                                                                                                                                                              \\ \hline
		Hound - Arya      &       & {[}(Hound - Arya){]}                                                                                                                                                                                                                                                                                                                                                                                                                                                                                 \\ \hline
		Eddard - Arya     &       & {[}(Hound - Arya),(Eddard - Arya){]}                                                                                                                                                                                                                                                                                                                                                                                                                                                                 \\ \hline
		Shae - Tyrion     &       & {[}(Jaime - Tyrion),(Shae - Tyrion){]}                                                                                                                                                                                                                                                                                                                                                                                                                                                               \\ \hline
		Joffrey - Robert  &       & {[}(Joffrey - Robert){]}                                                                                                                                                                                                                                                                                                                                                                                                                                                                             \\ \hline
		Samwell - Jon     &       & {[}(Jon - Jeor),(Samwell - Jeor),(Samwell - Jon){]}                                                                                                                                                                                                                                                                                                                                                                                                                                                  \\ \hline
		Cersei - Robert   & x     & {[}(Cersei - Mountain),(Joffrey - Robert), (Cersei - Robert){]}                                                                                                                                                                                                                                                                                                                                                                                                                                      \\ \hline
		Joffrey - Cersei  &       & \begin{tabular}[c]{@{}l@{}}{[}(Cersei - Mountain),(Joffrey - Robert), (Cersei - Robert),\\ (Joffrey - Cersei){]}\end{tabular}                                                                                                                                                                                                                                                                                                                                                                        \\ \hline
		Catelyn - Arya    & x     & \begin{tabular}[c]{@{}l@{}}{[}(Catelyn - Sansa),(Hound - Arya),(Eddard - Arya),\\ (Catelyn - Arya){]}\end{tabular}                                                                                                                                                                                                                                                                                                                                                                                   \\ \hline
		Eddard - Sansa    &       & \begin{tabular}[c]{@{}l@{}}{[}(Catelyn - Sansa),(Hound - Arya),(Eddard - Arya),\\ (Catelyn - Arya),(Eddard - Sansa){]}\end{tabular}                                                                                                                                                                                                                                                                                                                                                                  \\ \hline
		Sansa - Arya      &       & \begin{tabular}[c]{@{}l@{}}{[}(Catelyn - Sansa),(Hound - Arya),(Eddard - Arya),\\ (Catelyn - Arya),(Eddard - Sansa),(Sansa - Arya){]}\end{tabular}                                                                                                                                                                                                                                                                                                                                                   \\ \hline
		Eddard - Catelyn  &       & \begin{tabular}[c]{@{}l@{}}{[}(Catelyn - Sansa),(Hound - Arya),(Eddard - Arya),(Catelyn - Arya),\\ (Eddard - Sansa),(Sansa - Arya),(Eddard - Catelyn){]}\end{tabular}                                                                                                                                                                                                                                                                                                                                \\ \hline
		Sansa - Baelish   & x     & \begin{tabular}[c]{@{}l@{}}{[}(Varys - Baelish),(Sansa - Baelish),(Catelyn - Sansa),\\ (Hound - Arya),(Eddard - Arya),(Catelyn - Arya),\\ (Eddard - Sansa),(Sansa - Arya),(Eddard - Catelyn){]}\end{tabular}                                                                                                                                                                                                                                                                                         \\ \hline
		Catelyn - Baelish &       & \begin{tabular}[c]{@{}l@{}}{[}(Varys - Baelish),(Sansa - Baelish),(Catelyn - Sansa),\\ (Hound - Arya),(Eddard - Arya),(Catelyn - Arya),\\ (Eddard - Sansa),(Sansa - Arya),(Eddard - Catelyn),\\ (Catelyn - Baelish){]}\end{tabular}                                                                                                                                                                                                                                                                  \\ \hline
		Jaime - Cersei    & x     & \begin{tabular}[c]{@{}l@{}}{[}(Jaime - Tyrion),(Shae - Tyrion),(Cersei - Mountain),\\ (Joffrey - Robert), (Cersei - Robert),(Joffrey - Cersei){]}\end{tabular}                                                                                                                                                                                                                                                                                                                                       \\ \hline
		Cersei - Tyrion   &       & \begin{tabular}[c]{@{}l@{}}{[}(Jaime - Tyrion),(Shae - Tyrion),(Cersei - Mountain),\\ (Joffrey - Robert), (Cersei - Robert),(Joffrey - Cersei),\\ (Cersei - Tyrion){]}\end{tabular}                                                                                                                                                                                                                                                                                                                  \\ \hline
		
		
			\end{tabular}
	\end{table}

\begin{table}[H]
	\centering
	\begin{tabular}{|l|l|l|}
		\hline
		Link              & Merge & Graph                                                                                                                                                                                                                                                                                                                                                                                                                                                                                                \\ \hline

		Hound - Mountain  & x     & \begin{tabular}[c]{@{}l@{}}{[}(Varys - Baelish),(Sansa - Baelish),(Catelyn - Sansa),\\ (Hound - Arya),(Eddard - Arya),(Catelyn - Arya),\\ (Eddard - Sansa),(Sansa - Arya),(Eddard - Catelyn),\\ (Catelyn - Baelish),(Jaime - Tyrion),(Shae - Tyrion),\\ (Cersei - Mountain),(Joffrey - Robert), (Cersei - Robert),\\ (Joffrey - Cersei),(Cersei - Tyrion){]}\end{tabular}                                                                                                                            \\ \hline
		Joffrey - Jaime   &       & \begin{tabular}[c]{@{}l@{}}{[}(Varys - Baelish),(Sansa - Baelish),(Catelyn - Sansa),\\ (Hound - Arya),(Eddard - Arya),(Catelyn - Arya),\\ (Eddard - Sansa),(Sansa - Arya),(Eddard - Catelyn),\\ (Catelyn - Baelish),(Jaime - Tyrion),(Shae - Tyrion),\\ (Cersei - Mountain),(Joffrey - Robert), (Cersei - Robert),\\ (Joffrey - Cersei),(Cersei - Tyrion),(Joffrey - Jaime){]}\end{tabular}                                                                                                          \\ \hline
		Eddard - Jon      & x     & \begin{tabular}[c]{@{}l@{}}{[}(Varys - Baelish),(Sansa - Baelish),(Catelyn - Sansa),\\ (Hound - Arya),(Eddard - Arya),(Catelyn - Arya),\\ (Eddard - Sansa),(Sansa - Arya),(Eddard - Catelyn),\\ (Catelyn - Baelish),(Jaime - Tyrion),(Shae - Tyrion),\\ (Cersei - Mountain),(Joffrey - Robert), (Cersei - Robert),\\ (Joffrey - Cersei),(Cersei - Tyrion),(Joffrey - Jaime),\\ (Jon - Jeor),(Samwell - Jeor),(Samwell - Jon){]}\end{tabular}                                                         \\ \hline
		Eddard - Robert   &       & \begin{tabular}[c]{@{}l@{}}{[}(Varys - Baelish),(Sansa - Baelish),(Catelyn - Sansa),\\ (Hound - Arya),(Eddard - Arya),(Catelyn - Arya),\\ (Eddard - Sansa),(Sansa - Arya),(Eddard - Catelyn),\\ (Catelyn - Baelish),(Jaime - Tyrion),(Shae - Tyrion),\\ (Cersei - Mountain),(Joffrey - Robert), (Cersei - Robert),\\ (Joffrey - Cersei),(Cersei - Tyrion),(Joffrey - Jaime),\\ (Jon - Jeor),(Samwell - Jeor),(Samwell - Jon),\\ (Eddard - Robert){]}\end{tabular}                                    \\ \hline
		Joffrey - Hound   &       & \begin{tabular}[c]{@{}l@{}}{[}(Varys - Baelish),(Sansa - Baelish),(Catelyn - Sansa),\\ (Hound - Arya),(Eddard - Arya),(Catelyn - Arya),\\ (Eddard - Sansa),(Sansa - Arya),(Eddard - Catelyn),\\ (Catelyn - Baelish),(Jaime - Tyrion),(Shae - Tyrion),\\ (Cersei - Mountain),(Joffrey - Robert), (Cersei - Robert),\\ (Joffrey - Cersei),(Cersei - Tyrion),(Joffrey - Jaime),\\ (Jon - Jeor),(Samwell - Jeor),(Samwell - Jon),\\ (Eddard - Robert),(Joffrey - Hound){]}\end{tabular}                  \\ \hline
		Tyrion - Sansa    &       & \begin{tabular}[c]{@{}l@{}}{[}(Varys - Baelish),(Sansa - Baelish),(Catelyn - Sansa),\\ (Hound - Arya),(Eddard - Arya),(Catelyn - Arya),\\ (Eddard - Sansa),(Sansa - Arya),(Eddard - Catelyn),\\ (Catelyn - Baelish),(Jaime - Tyrion),(Shae - Tyrion),\\ (Cersei - Mountain),(Joffrey - Robert), (Cersei - Robert),\\ (Joffrey - Cersei),(Cersei - Tyrion),(Joffrey - Jaime),\\ (Jon - Jeor),(Samwell - Jeor),(Samwell - Jon),\\ (Eddard - Robert),(Joffrey - Hound),(Tyrion - Sansa){]}\end{tabular} \\ \hline
	\end{tabular}
\end{table}

%TODO draw dendogram, include listings


\begin{figure}[H]
	\centering
	\includegraphics[width=0.9\linewidth]{dendogram}
	\caption{Dendogram}
	\label{fig:dendogram}
\end{figure}

\item Visualisation of the communities

\begin{figure}[H]
	\centering
	\includegraphics[width=0.7\linewidth]{Inkedviz_LI}
	\caption{Visualisation of the network}
	\label{fig:viz}
\end{figure}

Here we can identify many communities. The two biggest are the ones we can have by separating the network in the middle (\textcolor{darkpastelgreen}{In green}). All the nodes have more links inside the communities than outside. 

\begin{table}[H]
	\centering
	\caption{The two big communities of the network separated by the \textcolor{darkpastelgreen}{green} line. All the nodes not cited below have $k_{out} = 0$. }
	\label{my-borness}
	\begin{tabular}{|l|l|l|}
		\hline
		\textbf{Node}      & $k_{in}$ & $k_{out}$ \\ \hline \hline
		Arya      & 3        & 1         \\ \hline
		Eddard    & 3        & 1         \\ \hline
		Sansa     & 3        & 1         \\ \hline \hline
		The Hound & 2        & 1         \\ \hline
		Robert    & 2        & 1         \\ \hline
		Tyrion    & 3        & 1         \\ \hline
	\end{tabular}
\end{table}

Here is two disjointed examples: 

\begin{table}[H]
	\centering
	\caption{Stark community in \textcolor{spirodiscoball}{light blue}. Each member have a $k_{in}$ bigger than the $k_{out}$ so the strong criterion applies. }
	\label{my-asdf}
	\begin{tabular}{|l|l|l|}
		\hline
		\textbf{Node}    & $k_{in}$ & $k_{out}$ \\ \hline
		Catelyn & 3        & 1         \\ \hline
		Arya    & 3        & 1         \\ \hline
		Eddard  & 3        & 2         \\ \hline
		Sansa   & 3        & 2         \\ \hline
	\end{tabular}
\end{table}

\begin{table}[H]
	\centering
	\caption{Jon's community in \textcolor{smalt}{dark blue}, the strong criterion applies too. }
	\label{my-label}
	\begin{tabular}{|l|l|l|}
		\hline
		Node    & $k_{in}$ & $k_{out}$ \\ \hline
		Jon     & 2        & 1         \\ \hline
		Jeor    & 2        & 0         \\ \hline
		Samwell & 2        & 0         \\ \hline
	\end{tabular}
\end{table}

\end{enumerate}

\end{document}