\documentclass[10pt,a4paper]{article}
\usepackage[utf8]{inputenc}

% \usepackage{ngerman}  % german documents
\usepackage{graphicx}  % import graphics einbinden
\usepackage{listings}  % support source code listing
\usepackage{amsmath}  % math stuff
\usepackage{amssymb} % 
\usepackage{a4wide} % wide pages
\usepackage{fancyhdr} % nice headers
\usepackage{float}
\usepackage{longtable}
\usepackage{xcolor}
\usepackage{booktabs}
\definecolor{darkpastelgreen}{rgb}{0.01, 0.75, 0.24}
\definecolor{spirodiscoball}{rgb}{0.06, 0.75, 0.99}
\definecolor{smalt}{rgb}{0.0, 0.2, 0.6}
\definecolor{armygreen}{rgb}{0.29, 0.33, 0.13}
\definecolor{awesome}{rgb}{1.0, 0.13, 0.32}
\definecolor{bittersweet}{rgb}{1.0, 0.44, 0.37}
\definecolor{bananayellow}{rgb}{1.0, 0.88, 0.21}
\definecolor{blue}{rgb}{0.0, 0.0, 1.0}
\definecolor{red}{rgb}{1.0, 0.0, 0.0}
\definecolor{green}{rgb}{0.0, 1.0, 0.0}



\lstset{basicstyle=\footnotesize,language=Python,breaklines=true,numbers=left, numberstyle=\tiny, stepnumber=5,firstnumber=0, numbersep=5pt} % set up listings
\pagestyle{fancy}             % header
\setlength{\parindent}{0pt}   % no indentation

\usepackage[pdfpagemode=None, colorlinks=true,  % url coloring
           linkcolor=blue, urlcolor=blue, citecolor=blue, plainpages=false, 
           pdfpagelabels,unicode]{hyperref}
           
% change enums style: first level (a), (b), (c)           
\renewcommand{\labelenumi}{(\alph{enumi})}
\renewcommand{\labelenumii}{(\arabic{enumii})}

\newcommand{\norm}[1]{\left\lVert#1\right\rVert}

%lecture name
\newcommand{\lecture}{
	Bioinformatics III
}           

%assignment iteration
\newcommand{\assignment}{
	Fifth Assignment
}


%set up names, matricle number, and email
\newcommand{\authors}{
  \begin{tabular}{rl}
    \href{mailto:s8tbscho@stud.uni-saarland.de}{Thibault Schowing} & (2571837)\\
    \href{mailto:wiebkeschmitt@outlook.de}{Wiebke Schmitt} & (2543675)
  \end{tabular}
}

% use to start a new exercise
\newcommand{\exercise}[1]
{
  \stepcounter{subsection}
  \subsection*{Exercise \thesubsection: #1}

}

\begin{document}
\title{\Large \lecture \\ \textbf{\normalsize \assignment}}
\author{\authors}

\setlength \headheight{25pt}
\fancyhead[R]{\begin{tabular}{r}\lecture \\ \assignment \end{tabular}}
\fancyhead[L]{\authors}


\setcounter{section}{5} % modify for later sheets, i.e. 2, 3, ...
%\section{Introduction to Python and some Network Properties} % optional, note that section invocation sets the section counter + 1, so adapt the setcounter command
\maketitle

%WIEBKE !!! READ THIS !!! 
% Usefull with Latex and maths: a list of the symbols ! https://reu.dimacs.rutgers.edu/Symbols.pdf

%EXERCICE 1
\exercise{ asdf }
\begin{enumerate}

% A
\item \textit{asdf}\\



% B
\item \textit{asdf}





% C
\item \textit{asdf}



\end{enumerate}


% NEW EXERCICE
\newpage
\exercise{aaas}
\begin{enumerate}
	
	% A
	\item \textit{Adding annotations to PPI-networks}\\
	
		%TODO listing at the end
		The listings for this exercise are at the end of the document.

		%\lstinputlisting[label=lst-1, caption={layout\string_main.py}] {../Scripts/layout\string_main.py}

	\item \textit{Generating an overview}\\
	
	
	
	
	
	
	For Chicken: 
	
	\begin{table}[H]
		\centering
		\caption{Chicken network overview}
		\label{chickentableoverview}
		\begin{tabular}{llllll}
			\hline
			\multicolumn{1}{|l|}{Interactions in the network}     & \multicolumn{1}{l|}{300} & \multicolumn{1}{l|}{}                           & \multicolumn{1}{l|}{}    & \multicolumn{1}{l|}{}               & \multicolumn{1}{l|}{}     \\ \hline
			\multicolumn{1}{|l|}{Proteins in the network}         & \multicolumn{1}{l|}{281} & \multicolumn{1}{l|}{Protein without annotation} & \multicolumn{1}{l|}{44}  & \multicolumn{1}{l|}{Percentage}     & \multicolumn{1}{l|}{15.6} \\ \hline
			\multicolumn{1}{|l|}{\textbf{Annotation per protein}} & \multicolumn{1}{l|}{}    & \multicolumn{1}{l|}{}                           & \multicolumn{1}{l|}{}    & \multicolumn{1}{l|}{}               & \multicolumn{1}{l|}{}     \\ \hline
			\multicolumn{1}{|l|}{Smallest number}                 & \multicolumn{1}{l|}{0}   & \multicolumn{1}{l|}{Average number}             & \multicolumn{1}{l|}{7.7} & \multicolumn{1}{l|}{Biggest number} & \multicolumn{1}{l|}{88}   \\ \hline
			\multicolumn{1}{|l|}{\textbf{Protein per annotation}} & \multicolumn{1}{l|}{}    & \multicolumn{1}{l|}{}                           & \multicolumn{1}{l|}{}    & \multicolumn{1}{l|}{}               & \multicolumn{1}{l|}{}     \\ \hline
			\multicolumn{1}{|l|}{Smallest number}                 & \multicolumn{1}{l|}{1}   & \multicolumn{1}{l|}{Average number}             & \multicolumn{1}{l|}{1.55} & \multicolumn{1}{l|}{Biggest number} & \multicolumn{1}{l|}{27}  \\ \hline
			&                          &                                                 &                          &                                     &                          
		\end{tabular}
	\end{table}

	For pig: 
	
	\begin{table}[H]
		\centering
		\caption{Pig network overview}
		\label{pignetworkoverview}
		\begin{tabular}{llllll}
			\hline
			\multicolumn{1}{|l|}{Interactions in the network}     & \multicolumn{1}{l|}{50} & \multicolumn{1}{l|}{}                           & \multicolumn{1}{l|}{}    & \multicolumn{1}{l|}{}               & \multicolumn{1}{l|}{}     \\ \hline
			\multicolumn{1}{|l|}{Proteins in the network}         & \multicolumn{1}{l|}{51} & \multicolumn{1}{l|}{Protein without annotation} & \multicolumn{1}{l|}{13}  & \multicolumn{1}{l|}{Percentage}     & \multicolumn{1}{l|}{25.5} \\ \hline
			\multicolumn{1}{|l|}{\textbf{Annotation per protein}} & \multicolumn{1}{l|}{}   & \multicolumn{1}{l|}{}                           & \multicolumn{1}{l|}{}    & \multicolumn{1}{l|}{}               & \multicolumn{1}{l|}{}     \\ \hline
			\multicolumn{1}{|l|}{Smallest number}                 & \multicolumn{1}{l|}{0}  & \multicolumn{1}{l|}{Average number}             & \multicolumn{1}{l|}{5.5} & \multicolumn{1}{l|}{Biggest number} & \multicolumn{1}{l|}{40}   \\ \hline
			\multicolumn{1}{|l|}{\textbf{Protein per annotation}} & \multicolumn{1}{l|}{}   & \multicolumn{1}{l|}{}                           & \multicolumn{1}{l|}{}    & \multicolumn{1}{l|}{}               & \multicolumn{1}{l|}{}     \\ \hline
			\multicolumn{1}{|l|}{Smallest number}                 & \multicolumn{1}{l|}{1}  & \multicolumn{1}{l|}{Average number}             & \multicolumn{1}{l|}{1.13} & \multicolumn{1}{l|}{Biggest number} & \multicolumn{1}{l|}{5} \\ \hline
			&                         &                                                 &                          &                                     &                          
		\end{tabular}
	\end{table}
	
	
	for Human: 
	
	
	\begin{table}[H]
		\centering
		\caption{Human network overview}
		\label{humanoverview}
		\begin{tabular}{llllll}
			\hline
			\multicolumn{1}{|l|}{Interactions in the network}     & \multicolumn{1}{l|}{275472} & \multicolumn{1}{l|}{}                           & \multicolumn{1}{l|}{}      & \multicolumn{1}{l|}{}               & \multicolumn{1}{l|}{}     \\ \hline
			\multicolumn{1}{|l|}{Proteins in the network}         & \multicolumn{1}{l|}{17087}  & \multicolumn{1}{l|}{Protein without annotation} & \multicolumn{1}{l|}{2262}  & \multicolumn{1}{l|}{Percentage}     & \multicolumn{1}{l|}{13.2} \\ \hline
			\multicolumn{1}{|l|}{\textbf{Annotation per protein}} & \multicolumn{1}{l|}{}       & \multicolumn{1}{l|}{}                           & \multicolumn{1}{l|}{}      & \multicolumn{1}{l|}{}               & \multicolumn{1}{l|}{}     \\ \hline
			\multicolumn{1}{|l|}{Smallest number}                 & \multicolumn{1}{l|}{0}      & \multicolumn{1}{l|}{Average number}             & \multicolumn{1}{l|}{7.22}  & \multicolumn{1}{l|}{Biggest number} & \multicolumn{1}{l|}{184}  \\ \hline
			\multicolumn{1}{|l|}{\textbf{Protein per annotation}} & \multicolumn{1}{l|}{}       & \multicolumn{1}{l|}{}                           & \multicolumn{1}{l|}{}      & \multicolumn{1}{l|}{}               & \multicolumn{1}{l|}{}     \\ \hline
			\multicolumn{1}{|l|}{Smallest number}                 & \multicolumn{1}{l|}{1}      & \multicolumn{1}{l|}{Average number}             & \multicolumn{1}{l|}{10.6} & \multicolumn{1}{l|}{Biggest number} & \multicolumn{1}{l|}{1554} \\ \hline
			&                             &                                                 &                            &                                     &                          
		\end{tabular}
	\end{table}

	\newpage
	\item \textit{Examining the most/least common annotations}
	
	
	
	
		% Please add the following required packages to your document preamble:
		% \usepackage{booktabs}
		% \usepackage{graphicx}
		\begin{table}[H]
			\centering
			\caption{Function of the 5 most common GO identifiers of the human network. }
			\label{my-label}
				\begin{tabular}{|l|l|l|}
					\hline
					\textbf{GO id} & \textbf{Quantity} & \textbf{Biological Process}                                                                                                                                           \\ \hline
					GO:0006351     & 1562              & The cellular synthesis of RNA on a template of DNA.                                                                                                                   \\ \hline
					GO:0045944     & 1029              & \begin{tabular}[c]{@{}l@{}}Any process that activates or increases the frequency, \\ rate or extent of transcription from an RNA polymerase II promoter.\end{tabular} \\ \hline
					GO:0007165     & 1010              & Signal transduction                                                                                                                                                   \\ \hline
					GO:0006357     & 960               & \begin{tabular}[c]{@{}l@{}}Any process that modulates the frequency, rate or extent \\ of transcription mediated by RNA polymerase II.\end{tabular}                   \\ \hline
					GO:0006355     & 765               & \begin{tabular}[c]{@{}l@{}}Any process that modulates the frequency, rate or extent \\ of cellular DNA-templated transcription\end{tabular}                           \\ \hline
				\end{tabular}
		\end{table}
	
	We can observe that these annotations concerns general process happening almost in every cell. This explains why they are the most common in opposition as the annotations in the table below, which concerns specific reaction or process concerning particular location on molecules. 
	
	
	
	\begin{table}[H]
		\centering
		\caption{Function of the 5 least common GO identifiers of the human network}
		\label{my-label}
		\begin{tabular}{|l|l|l|}
			\hline
			\textbf{GO id} & \textbf{Quantity} & \textbf{Biological Process}                 \\ \hline
			GO:0000003     & 1                 & Reproduction                                \\ \hline
			GO:0000011     & 1                 & Vacuole inheritance                         \\ \hline
			GO:0000032     & 1                 & Cell wall mannoprotein biosynthetic process \\ \hline
			GO:0000053     & 1                 & Argininosuccinate metabolic process         \\ \hline
			GO:0000097     & 1                 & Sulfur amino acid biosynthetic process      \\ \hline
		\end{tabular}
	\end{table}


	\item \textit{Investigating annotation enrichment}
	
	%TODO complete table 
	\begin{table}[H]
		\centering
		\caption{My caption}
		\label{my-label}
		\begin{tabular}{|l|l|l|}
			\hline
			& Number & Percentage \\ \hline
			p \textless 0.05    &        &            \\ \hline
			p \textgreater 0.5  &        &            \\ \hline
			p \textgreater 0.95 &        &            \\ \hline
		\end{tabular}
	\end{table}

	


	%TODO ask for result credibility
	
	\begin{table}[H]
		\centering
		\caption{Annotations with the five lowest $pA$ and five highest $pA$  }
		\label{my-label}
		\begin{tabular}{|l|l|l|l|l|}
			\hline
			\textbf{GO:ID }     & \textbf{pA}  & \textbf{Nb Protein} & \begin{tabular}[c]{@{}l@{}}\textbf{Nb Interact.} \\ \textbf{protein}\end{tabular} & \textbf{Annotation}   \\ \hline
			GO:0009409 & 4.3907e-07  & 3 &    3     & Response to cold   \\ \hline
			GO:0030154 & 1.7908e-05  & 7 &    4     & Cell differentiation   \\ \hline
			GO:0007169 & 0.0002 & 3 &    2     & \begin{tabular}[c]{@{}l@{}}Transmembrane receptor protein\\ tyrosine kinase signaling pathway\end{tabular} \\ \hline
			GO:0000712 & 0.0002 & 3 &    2     & \begin{tabular}[c]{@{}l@{}}Resolution of meiotic recombination\\ intermediates\end{tabular}                \\ \hline
			GO:0032570 & 0.0002 & 3 &    2     & Response to progesterone   \\ \hline
			
			GO:0007049 & 1                      & 10 &    0    & Cell cycle  \\ \hline
			GO:0006096 & 1                      & 9  &    0    & Glycotic process     \\ \hline
			GO:0055114 & 1                      & 9  &    0    & Oxydation-reduction process    \\ \hline
			GO:0006457 & 1                      & 9  &    0    & Protein folding    \\ \hline
			GO:0006094 & 1                      & 8  &    0    & Gluconeogenesis  \\ \hline
		\end{tabular}
	\end{table}
	
	\item \textit{e) Investigating annotation combinations}
	
	
	
	
\end{enumerate}

\end{document}