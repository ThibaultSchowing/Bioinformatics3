\documentclass[10pt,a4paper]{article}
\usepackage[utf8]{inputenc}

% \usepackage{ngerman}  % german documents
\usepackage{graphicx}  % import graphics einbinden
\usepackage{listings}  % support source code listing
\usepackage{amsmath}  % math stuff
\usepackage{amssymb} % 
\usepackage{a4wide} % wide pages
\usepackage{fancyhdr} % nice headers
\usepackage{float}
\usepackage{longtable}
\usepackage{xcolor}
\usepackage{booktabs}
\definecolor{darkpastelgreen}{rgb}{0.01, 0.75, 0.24}
\definecolor{spirodiscoball}{rgb}{0.06, 0.75, 0.99}
\definecolor{smalt}{rgb}{0.0, 0.2, 0.6}
\definecolor{armygreen}{rgb}{0.29, 0.33, 0.13}
\definecolor{awesome}{rgb}{1.0, 0.13, 0.32}
\definecolor{bittersweet}{rgb}{1.0, 0.44, 0.37}
\definecolor{bananayellow}{rgb}{1.0, 0.88, 0.21}
\definecolor{blue}{rgb}{0.0, 0.0, 1.0}
\definecolor{red}{rgb}{1.0, 0.0, 0.0}
\definecolor{green}{rgb}{0.0, 1.0, 0.0}

% for multiple figures
\usepackage{subcaption}


\lstset{basicstyle=\footnotesize,language=Python,breaklines=true,numbers=left, numberstyle=\tiny, stepnumber=5,firstnumber=0, numbersep=5pt} % set up listings
\pagestyle{fancy}             % header
\setlength{\parindent}{0pt}   % no indentation
 

\usepackage[pdfpagemode=None, colorlinks=true,  % url coloring
           linkcolor=blue, urlcolor=blue, citecolor=blue, plainpages=false, 
           pdfpagelabels,unicode]{hyperref}
           
% change enums style: first level (a), (b), (c)           
\renewcommand{\labelenumi}{(\alph{enumi})}
\renewcommand{\labelenumii}{(\arabic{enumii})}

\newcommand{\norm}[1]{\left\lVert#1\right\rVert}

%lecture name
\newcommand{\lecture}{
	Bioinformatics III
}           

%assignment iteration
\newcommand{\assignment}{
	Eighth Assignment
}


%set up names, matricle number, and email
\newcommand{\authors}{
  \begin{tabular}{rl}
    \href{mailto:s8tbscho@stud.uni-saarland.de}{Thibault Schowing} & (2571837)\\
    \href{mailto:wiebkeschmitt@outlook.de}{Wiebke Schmitt} & (2543675)
  \end{tabular}
}

% use to start a new exercise
\newcommand{\exercise}[1]
{
  \stepcounter{subsection}
  \subsection*{Exercise \thesubsection: #1}

}

\begin{document}
\title{\Large \lecture \\ \textbf{\normalsize \assignment}}
\author{\authors}

\setlength \headheight{25pt}
\fancyhead[R]{\begin{tabular}{r}\lecture \\ \assignment \end{tabular}}
\fancyhead[L]{\authors}


\setcounter{section}{8} % modify for later sheets, i.e. 2, 3, ...
%\section{Introduction to Python and some Network Properties} % optional, note that section invocation sets the section counter + 1, so adapt the setcounter command
\maketitle

%EXERCICE 1
\exercise{Data Preprocessing}

\begin{enumerate}

% A
\item \textbf{Data matrix:}\textit{The supplement contains the data\string_matrix.py–file with the outline of a
	DataMatrix–class in which you should complete. }\\ 

\lstinputlisting[label=qtew-1, caption={Data Matrix class script}] {../Scripts/data\string_matrix.py}


%B
\item \textbf{Process expression and methylation data:}\textit{In the function exercise 1() in main.py,
	use your DataMatrix–class to read in the expression and methylation tables given in the
	supplement and write the processed matrices into files.} \footnote{The files are attached with the source files in the email.}\\

\lstinputlisting[label=asdseef, caption={Main programm}] {../Scripts/main.py}


\textit{For each input file, report the number of genes and samples whose data does not follow a
	normal distribution with $\alpha$ = 0.05.
}\\

Number of genes whose data does not follow a normal distribution (EXPRESSION):  73\\

Number of sample whose data does not follow a normal distribution (EXPRESSION):  19\\

Number of genes whose data does not follow a normal distribution (METHYLATION):  66\\

Number of sample whose data does not follow a normal distribution (METHYLATION):  19\\



\end{enumerate}


%\newpage
%\lstinputlisting[label=qtew-1, caption={boolean\textunderscore network.py}] {../Scripts/boolean\string_network.py}
%\newpage
%\lstinputlisting[label=qtew-1, caption={Main function that tests the boolean network}] {../Scripts/main\string_boolean.py}




% NEW EXERCICE
\newpage
\exercise{Correlation Measures}
\lstinputlisting[label=qtew-1, caption={Correlation matrix}] {../Scripts/correlation.py}



% NEW EXERCICE
\newpage
\exercise{Gene Co–Expression Networks}
\begin{enumerate}
	
	% A
	\item \textbf{Network construction}\\
	
	\lstinputlisting[label=lsdddt-1, caption={Correlation network}] {../Scripts/network.py}
	
	
	% B
	\item \textbf{Network visualisation\footnote{The files are attached with the source files in the email.}}
	
	
\begin{figure}[H]
	\centering
	\includegraphics[width=0.9\linewidth]{img/schmitt_schowing_expression_network_kendall.png}
	\caption{Expression network with Kendall correlation}
	\label{fig:schmittschowingexpressionnetworkkendall}
\end{figure}

	
\begin{figure}[H]
	\centering
	\includegraphics[width=0.9\linewidth]{img/schmitt_schowing_expression_network_pearson.png}
	\caption{Expression network with Pearson correlation}
	\label{fig:schmittschowingexpressionnetworkpearson}
\end{figure}

	
\begin{figure}[H]
	\centering
	\includegraphics[width=0.9\linewidth]{img/schmitt_schowing_expression_network_spearman.png}
	\caption{Expression network with Spearman correlation}
	\label{fig:schmittschowingexpressionnetworkspearman}
\end{figure}


\begin{figure}[H]
	\centering
	\includegraphics[width=0.9\linewidth]{img/schmitt_schowing_methylation_network_kendall}
	\caption{Methylation network with Kendall correlation}
	\label{fig:schmittschowingmethylationnetworkkendall}
\end{figure}



\begin{figure}[H]
	\centering
	\includegraphics[width=0.9\linewidth]{img/schmitt_schowing_methylation_network_pearson}
	\caption{Methylation network with Pearson correlation}
	\label{fig:schmittschowingmethylationnetworkpearson}
\end{figure}


\begin{figure}[H]
	\centering
	\includegraphics[width=0.9\linewidth]{img/schmitt_schowing_methylation_network_spearman}
	\caption{Methylation network with Spearman correlation}
	\label{fig:schmittschowingmethylationnetworkspearman}
\end{figure}




	% C
	\item \textbf{Discussion:}\textit{Briefly comment on the similarities and difference between the networks. Explain and discuss your results.}\\
	
	We observe that the number of highly correlated gene is much higher when we look at the methylation in opposition as when we look at the expression. 
	
	
\end{enumerate}

% NEW EXERCICE
\newpage
\exercise{Hierarchical Clustering}
\begin{enumerate}
	
	% A
	\item \textbf{Implementation: }
	
	\lstinputlisting[label=lsdddt-1, caption={Hierarchical clustering}] {../Scripts/cluster.py}
	
	\item \textbf{Application:}\textit{In the function exercise 4() in main.py (listing \ref{asdseef}), use your implementation to hierarchically cluster the expression and methylation data tables with the Pearson, Spearman and Kendall correlation coefficient. This should give you a total of 6 TSV files}\\
	
	
	\item \textbf{Discussion: }\textit{Can hierarchical clustering be used to differentiate between blood cells and skin tissues? Are there differences between the correlation coefficients or data type? Why?}\\
	
	Let's first recall the two different types of cells. In this experiment we have the samples HSC, MPP1, MPP2, CLP, CMP, GMP, MEP, CD4, CD8, B cell, Eryth, Granu and Mono that are from blood cells, whereas the samples TBSC, ABSC, MTAC, CLDC, EPro and EDif from skin tissues. In the listing \ref{samersdfae} at line 15, one of the last merge, we can see that the two merged cluster are from samples from different cells (blood or skin) meaning that the clustering is working well\footnote{Notice that here, the ABSC sample is not present due to a too small correlation with the others, and therefore, is not present in the .}. We can assume here that the genes in a skin cell or in a blood cell are not expressed and/or methylated in the same way as the function of the cells are not the same. 
	
	\lstinputlisting[language={}, label=samersdfae, caption={Hierarchical clustering example with methylation data and Pearson correlation}] {../Scripts/schmitt\string_schowing\string_methylation\string_cluster\string_pearson.tsv}
	
	
	\begin{figure}[H]
		\centering
		\includegraphics[width=\paperwidth, angle=90]{img/clustering_methylation_pearson_dendrogram}
		\caption{Dendrogram of the clustering in listing \ref{samersdfae}. The blue nodes are skin cells, and the orange one blood cells.}
		\label{fig:clusteringmethylationpearsondendrogram}
	\end{figure}
	

	
\end{enumerate}

\end{document}